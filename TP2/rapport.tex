\documentclass[12pt,a4paper]{article}
\usepackage[utf8]{inputenc}
\usepackage[T1]{fontenc}
\usepackage[french]{babel} 
\usepackage{amsmath}
\usepackage{graphicx}
\usepackage{booktabs}
\usepackage{hyperref}
\usepackage{geometry}
\usepackage{minted}
\usepackage{textalpha}
\usepackage{array}
\usepackage{siunitx}
\usepackage{subcaption}

\geometry{margin=2.5cm}

\title{TP2 : Classification de caractéristiques locales \\
Approches supervisée (bayésienne) et non
supervisée (K-Means) \\ \large Cours CSC\_4MI04}
\author{Júlia Ellen Dias Leite \\ Helena Guachalla de Andrade \\ Mateus Bastos Soares}
\date{Février 2026}

\begin{document}

\maketitle

\section*{1. Principes de la classification bayésienne}

\subsection*{1.1 Principes généraux}

Quelle différence y a-t-il entre la classification par le critère du maximum de vraisemblance (ML) et celui du maximum a posteriori (MAP) ? Comment cette différence se traduira-t-elle dans vos programmes ?

\subsection*{1.2 Application à la classification bayésienne des pixels}

Quelles sont les limites de l'utilisation des eules couleurs comme espace d'observation (caractéristiques) ? Quelles autres caractéristiques pouvez-vous proposer pour la segmentaation d'images ?

\section*{2. Classification de caractéristiques locales}

\subsection*{2.1 Classification bayésienne des pixels}

Quelles vous semblent être les espaces d’observations et le nombre de classes les mieux adaptés à la tâche qui vous intéresse ? Quelles sont les difficultés potentielles ? Quelles différences observez-vous entre le modèle bayésien gaussien multidimensionnel et le modèle bayésien gaussien naïf ?

\subsection*{2.2 Classification non supervisée des pixels par K-Means}

Quelles vous semblent être les espaces d’observations et le nombre de classes les mieux adaptés à la tâche qui vous intéresse ? Quelles sont les difficultés potentielles ? Quels sont les avantages et les inconvénients par rapport à l’approche précédente ?

\newpage

\bibliographystyle{apalike}
\bibliography{bibliography.bib}

\end{document}
